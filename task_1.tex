\documentclass[]{article} 
\usepackage[fontsize=14pt]{fontsize} 
\usepackage{mathtools}
\usepackage{fontspec}
\setmainfont{CMU Serif}
\setsansfont{CMU Sans Serif}
\usepackage[% 
 a4paper,% 
 footskip=1cm,% 
 headsep=0.3cm,% 
 top=2cm, %поле сверху 
 bottom=2cm, %поле снизу 
 left=2cm, %поле ліворуч 
 right=2cm, %поле праворуч 
    ]{geometry} 
 
\usepackage{microtype} 
\usepackage{amsthm}
\usepackage{logix} 
 
\usepackage[english, ukrainian]{babel} 
\usepackage{indentfirst} 
\theoremstyle{plain} 
\newtheorem{lemma}{Лема}
\setlength{\jot}{3pt} 

\title{\sffamily Рівневий набір гармонійних функцій} 
\author{А. В. Тор} 
\date{}
 
\renewcommand{\baselinestretch}{1.5} 
\begin{document} 
\maketitle
Для $\theta \in [0, \pi/2[$, розглянемо множини
\begin{align*}
    \Sigma_{1,\theta} &= \biggl\{ a \in \mathbb{C} \setminus ]{-\infty}, -1] : \Re \biggl( \int_{[1,a]} e^{i\theta} \sqrt{p_a(z)} \; dz \biggr) = 0 \biggr\}; \\
    \Sigma_{-1,\theta} &= \biggl\{ a \in \mathbb{C} \setminus [1, +\infty[ : \Re \biggl( \int_{[-1,a]} e^{i\theta} \sqrt{p_a(z)} \; dz \biggr) = 0 \biggr\}; \\
    \Sigma_{\theta} &= \biggl\{ a \in \mathbb{C} \setminus [-1, 1] : \Re \biggl( \int_{[-1,1]} e^{i\theta} \sqrt{p_a(z)} \; dz \biggr) = 0 \biggr\};
\end{align*}
де $p_a(z)$ \textemdash\ комплексний многочлен, визначений формулою
\begin{equation*}
    p_a(z) = (z - a)(z^2 - 1).
\end{equation*}

\begin{lemma}
Нехай $\theta \in [0, \pi/2[$. Тоді кожна з множин $\Sigma_{1,\theta}$ та $\Sigma_{-1,\theta}$ утворюється двома гладкими кривими, які локально ортогональні відповідно при $z = 1$ та $z = -1$, точніше:
\begin{align*}
    \lim_{a \to -1, \, a \in \Sigma_{-1,\theta}} \arg(a + 1) =\frac{-2\theta + (2k + 1) \pi}{4}, \quad k = 0, 1, 2, 3; \\ 
    \lim_{a \to +1, \, a \in \Sigma_{1,\theta}} \arg(a - 1) = \frac{-\theta + k\pi}{2}, \quad k = 0, 1, 2, 3.
\end{align*}

Дві криві, що визначають $\Sigma_{1,\theta}$ (відповідно $\Sigma_{-1,\theta}$), перетинаються лише при $z = 1$ (відповідно $z = -1$). Більше того, для $\theta \notin \{0, \pi/2\}$, вони розходяться по-різному до $\infty$ в одному з напрямків
\begin{equation*}
    \lim_{|a| \to +\infty, \, a \in \Sigma_{\pm 1, \theta}} \arg a = \frac{-2\theta + 2k\pi}{5}, \quad k = 0, 1, 2, 3, 4.
\end{equation*}
Для $\theta = 0$, (відповідно $\theta = \pi/2$), один промінь $\Sigma_{1,\theta}$ (відповідно $\Sigma_{-1,\theta}$) розходиться до $z = -1$ (відповідно $z = 1$).
\end{lemma}
\begin{proof}
Нехай задано непостійну гармонічну функцію $u$, визначену в деякій області $\mathcal{D}$ of $\mathbb{C}$. Критичними точками $u$ є саме ті, де
\begin{equation*}
    \frac{\partial u}{\partial z} = \frac{1}{2} \left( \frac{\partial u}{\partial x} - i \frac{\partial u}{\partial y} \right) = 0.
\end{equation*}
Вони ізольовані. Якщо $v$ є гармонічним спряженням $u$ у $\mathcal{D}$, скажімо, $f(z) = u(z) + iv(z)$ аналітична у $\mathcal{D}$, тоді за Коші-Ріманом,
\begin{equation*}
    f'(z) = 0 \iff u'(z) = 0.
\end{equation*}
Встановлений рівень
\begin{equation*}
    \Sigma_{z_0} = \{ z \in \mathcal{D} : u(z) = u(z_0) \}
\end{equation*}
$u$ через точку $z_0 \in \mathcal{D}$ залежить від поведінки $f$ поблизу $z_0$. Точніше, якщо $z_0$ є критичною точкою $u$, ($u'(z_0) = 0$), то існує околиця $\mathcal{U}$ околу $z_0$, голоморфної функції $g(z)$, визначена на $\mathcal{U}$, така, що
\begin{equation*}
    \forall z \in \mathcal{U}, \quad f(z) = (z - z_0)^m g(z), \quad g(z) \neq 0.
\end{equation*}
Взявши гілку $m$-го кореня з $g(z)$, $f$ має локальну структуру
\begin{equation*}
     f(z) = (h(z))^m, \quad \forall z \in U.
\end{equation*}
Звідси випливає, що $\Sigma_{z_0}$ локально утворена $m$ аналітичними дугами які проходять через $z_0$ і перетинаються там під рівними кутами $\pi/m$. Через регулярну точку $z_0 \in \mathcal{D}$, ($u'(z_0) \neq 0$), теорема про неявну функцію стверджує, що $\Sigma_{z_0}$ є локально єдиною аналітичною дугою. Зауважте, що множина рівнів гармонічної функції не може закінчуватися у звичайній точці.

Розглянемо багатозначну функцію
\begin{equation*}
    f_{1,\theta}(a) = \int_{1}^{a} e^{i\theta} \sqrt{p_a(t)} \; dt, \quad a \in \mathbb{C}. 
\end{equation*}
Інтегруючи вздовж відрізка $[1, a]$, можна припустити, що без без втрати загальності, що
\begin{equation}\label{eq:formula1}
    f_{1,\theta}(a) = i e^{i\theta} (a - 1)^2 
\int_{0}^{1} \sqrt{t(1 - t)} \sqrt{t(a - 1) + 2} \; dt = (a - 1)^2 g(a), \quad g(1) \neq 0.
\end{equation}
Очевидно, що:
\begin{equation*}
    \forall a \in \mathbb{C} \setminus ]{-\infty}, -1], \quad \{ t(a - 1) + 2 : t \in [0, 1] \} = [2, a + 1] \subset \mathbb{C} \setminus ]{-\infty}, 0].
\end{equation*}
Отже, при фіксованому виборі аргументу та квадратного кореня всередині інтеграла, $f_{1,\theta}$ та $g$ є однозначними аналітичними функціями в $\mathbb{C} \setminus ]{-\infty}, -1]$.

Припустимо, що для деяких $a \in \mathbb{C} \setminus ]{-\infty}, -1]$, $a \neq -1$,
\begin{equation*}
    u(a) = \Re f_{1,\theta}(a) = 0, \quad f'_{1,\theta}(a) = 0. 
\end{equation*}
Тоді,
\begin{equation*}
    (a - 1)^3 g'(a) + 2f_{1,\theta}(a) = 0.
\end{equation*}
Беручи справжні деталі, ми отримуємо
\begin{align*}
     0 = \int_{0}^{1} \sqrt{t(1 - t)} 
\Im \left( e^{i\theta} (a - 1)^2 \sqrt{t(a - 1) + 2} \right) dt,\\
0 = \Re \left( (a - 1)^3 g'(a) \right) = \int_{0}^{1} t \sqrt{t(1 - t)} 
\Im \left( e^{i\theta} (a - 1)^3 \sqrt{t(a - 1) + 2} \right) dt.
\end{align*}

За неперервністю функцій всередині цих інтегралів на відрізку $[0, 1]$, існують $t_1, t_2 \in [0, 1]$ такі, що
\begin{equation*}
     \Im \left( e^{i\theta} (a - 1)^2 \sqrt{t_1(a - 1) + 2} \right) = \Im \left( e^{i\theta} (a - 1)^3 \sqrt{t_2(a - 1) + 2} \right) = 0,
\end{equation*}
а потім
\begin{equation*}
    e^{2i\theta} (a - 1)^4 (t_1(a - 1) + 2) > 0, \quad e^{2i\theta} (a - 1)^6 (t_2(a - 1) + 2) > 0.
\end{equation*}
Взявши їх співвідношення, отримуємо
\begin{equation*}
     \frac{(t_1(a - 1) + 2)(t_2(a - 1) + 2)}{(a - 1)^2} > 0,
\end{equation*}
яка не може виконуватись, оскільки, якщо $\Im a > 0$, то
\begin{equation*}
     0 < \arg(t_1(a - 1) + 2) + \arg(t_2(a - 1) + 2) < 2 \arg(a + 1) < \arg((a - 1)^2) < 2\pi.
\end{equation*}
Випадок $\Im a < 0$ є аналогічним, тоді як випадок $a \in \mathbb{R}$ можна легко відкинути. Таким чином, $a = 1$ є єдиною критичною точкою $\Re f_{1,\theta}$. Since $f''_{1,\theta}(1) = 2g(1) \neq 0$, ми виводимо локальну поведінку $\Sigma_{1,\theta}$ поблизу $a = 1$.

Припустимо, що для деяких $\theta \in ]0, \pi/2[$, промінь $\Sigma_{\pm 1,\theta}$ розходиться до певного моменту в $]{-\infty}, -1[$; або наприклад,
\begin{equation*}
  (\Sigma_{1,\theta} \setminus \Sigma_{1,\theta}) \cap \{ z \in \mathbb{C} : \Im z \geq 0 \} = \{x_{\theta}\}.   
\end{equation*}
Нехай $\epsilon > 0$ так, що $0 < \theta - 2\epsilon$. Для $a \in \mathbb{C}$, що задовольняє $\pi - \epsilon < \arg a < \pi$,
\begin{equation*}
    0 < \theta - 2\epsilon < \theta + 2 \arg a + \arg \int_{0}^{1} \sqrt{t(1 - t)} \sqrt{t(a - 1) + 2} dt < \frac{\pi}{2} + \theta - \frac{\epsilon}{2} < \pi,
\end{equation*}
що суперечить \eqref{eq:formula1}. Інші випадки подібні. Таким чином, будь-який промінь з $\Sigma_{\pm 1,\theta}$ повинен розходитись на $\infty$. Випадок $\theta = 0$ є простішим.

Якщо $a \to \infty$, тоді $|f_{1,\theta}(a)| \to +\infty$; since $\Re f_{1,\theta}(a) = 0$, we have $|\Im f_{1,\theta}(a)| \to +\infty$. Звідси випливає, що
\begin{equation*}
    \arg(f(a)) \sim \arg \left( \frac{4}{15} e^{i\theta} a^{5/2} \right) \to \frac{\pi}{2} + k\pi, \quad k \in \mathbb{Z}, \quad \text{при } a \to \infty. 
\end{equation*}
Ми отримуємо поведінку будь-якої дуги $\Sigma_{1,\theta}$, яка розходиться до $\infty$. Зокрема, з принципу максимуму модуля, два промені з $\Sigma_{,\theta}$ не можуть розходитись у $\infty$. $\Sigma_{,\theta}$ не можуть розходитися до $\infty$ в одному напрямку. 

Якщо $\Sigma_{1,\theta}$ містить регулярну точку $z_0$ (наприклад, $\Im z_0 > 0$), яка не належить дугам $\Sigma_{1,\theta}$, що виходять з точки $a = 1$. Два промені кривої набору рівнів $\gamma$, що проходять через $z_0$ розходяться до $\infty$ у двох різних напрямках. Звідси випливає, що $\gamma$ має проходити через $z_1 = 1 + iy$, для деяких $y > 0$, або $z_1 = y$, для деяких $y > 1$. Легко перевірити, що в обох випадках, для будь-якого вибору аргументу,
\begin{equation*}
    \Re \int_{1}^{z_1} e^{i\theta} \sqrt{p_{z_1}(t)} dt \neq 0,
\end{equation*}
і отримуємо протиріччя. Таким чином, $\Sigma_{1,\theta}$ утворюється лише двома двома кривими. Що проходять через $a = 1$. Таку ж саму ідею дає структура  $\Sigma_{-1,\theta}$; навіть більше, з співвідношення
\begin{equation} \label{eq: formula2}
    \Re f_{\pm 1,\theta}(a) = 0 \iff \Re f_{\pm 1, \pi/2 - \theta}(-a) = 0,
\end{equation}
доступних для довільного $\theta \in [\pi/4, \pi/2 [,$ можна легко побачити, що $\Sigma_{-1, \pi/2 - \theta}$ і $\Sigma_{1, \theta}$ симетричні відносно уявної осі \eqref{eq: formula2}. Це приводить нас до того, щоб обмежити наше дослідження випадком.
\end{proof}
\end{document}
